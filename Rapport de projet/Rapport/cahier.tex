\documentclass{article}

\usepackage[svgnames]{xcolor}

\usepackage{xcolor}
\usepackage[utf8]{inputenc}
\usepackage[francais]{babel}
\usepackage{graphicx}

\usepackage{geometry}
\geometry{hmargin=2.5cm,vmargin=2.5cm}

\usepackage{fancyhdr}
\pagestyle{fancy}
\fancyhead[R]{Rapport de projet}
\fancyhead[L]{Aliénor Roussel, Laëtitia Viau, 
\newline Thibault Souvannavong, Abdellatif Ben Mlah}
\fancyfoot[C]{}

\renewcommand{\footrulewidth}{1pt}
\fancyfoot[C]{\textbf{Page \thepage}}

\renewcommand{\thesubsection}{\Alph{subsection}}
\renewcommand{\thesubsubsection}{\arabic{subsubsection}}

\setlength{\parindent}{0cm}
\setlength{\parskip}{1ex plus 0.5ex minus 0.2ex}
\newcommand{\hsp}{\hspace{20pt}}
\newcommand{\HRule}{\rule{\linewidth}{0.5mm}}

\begin{document}

\begin{titlepage}
  \begin{sffamily}
  \begin{center}

    \textsc{\LARGE Rapport de projet}\\[1cm]

    % Title
    \HRule \\[0.1cm]
 \includegraphics[scale=0.12]{logo.png}
    
    \HRule \\[1cm]
    \begin{center}
       \includegraphics[scale=2.2]{photo_couv.jpg}
          \\[0.5cm]
       \end{center} 
 
\Large{Aliénor Roussel, Laëtitia Viau, Thibault Souvannavong et Abdellatif Ben Mlah}

\vfill 
\vfill

    {\large Projet Licence 3 MIASHS 2017-2018}
\includegraphics[scale=0.2]{fac.png}
  \end{center}
  \end{sffamily}
\end{titlepage}

\tableofcontents

\newpage

\section{Introduction}

L'enjeux de notre projet est d'apporter une interface web permettant d’élaborer une liste d’activités à faire et des bons plans sur Montpellier et ses alentours. A travers ce site web, ses utilisateurs ont également la possibilité de proposer et/ou de rejoindre des activités de groupe. En effet, nous souhaitons offrir la possibilité aux utilisateurs de faire de nouvelles rencontres.\\
Pour cela, différents critères de recherche en fonction des préférences de l’utilisateur ont été mis en place.\\
Un utilisateur peut donc à la fois rechercher des idées d'activités et à la fois créer ou rejoindre des activités avec d'autres personnes.
Notre domaine d'application se limite à Montpellier et ses alentours afin de proposer des types d'activités variés dans le but de répondre aux demandes de chaque utilisateur.


\section{Gestion de projet}

\textbf{ Tableau de la contribution de chacun par tâche}

N'ayant pas encore eu les enseignements sur la création de page web dynamique au premier semestre nous avons fais une estimation de planning pour notre projet.
Il s'est avéré que ce planning a été grandement modifié. 
En effet, nous avons commencer la création des pages html et la création des feuilles de style CSS en même temps puisque il s'est avéré que les deux étaient liés.
De plus, la conception de la base de donnée a été plus longue que prévu du fait du manque de données : beaucoup d'informations étaient manquantes dans les données que nous avions récoltées au premier semestre. Nous avons donc nous y adapter.\\
La quantité et la complexité des tables que nous avons du mettre en place pour avoir au final une base claire et logique a également participé à la durée de la tâche.\\
Il a par ailleurs fallu s'adapter à des contre-temps tel que le blocage de la faculté et donc le manque de compétences dans certaines branches de programmation. 
Nous avons donc dû faire des choix pour rendre le projet à temps et n'avons pas pu mettre en place le système de localisation sur carte des activités. \\
Aussi, nous avons décider de remplacer la page bons plans par nos activités coups de coeur sur la page d'accueil.
Également nous avons dû chercher comment mettre en place certaines fonctionnalités du site sur internet à cause du retard dans certaines matières (JavaScript). 


\section{Description technique}
\begin{description}
\item \textbf{Liste des technologies utilisées}
\begin{itemize}
\item Git
\item Php
\item Javascript
\item Mamp
\item Bluefish
\item LaTex
\end{itemize}
\end{description}
Schéma de l'architecture de l'application
Schéma de la base de données

\section{Description fonctionnelle}

\begin{description}
\item \textbf{Liste des fonctionnalités de Move'n Meet}
\begin{itemize}
\item Connexion / Inscription sur le site
\item Création d'une activité avec un nombre de personnes maximum
\item Recherche d'une idée d'activité à travers des critères et des sous-critères
\item Parcourir la liste des évènements temporaires
\item Parcourir une liste de bons plans
\item Rejoindre une activité déjà crée après s'être connecté
\item Se désinscrire d'une activité 
\end{itemize}
\end{description}

\begin{description}
\item \textbf{Liste de gestion des erreurs mise en place}
\begin{itemize}
\item Un utilisateur ne peut pas s'inscrire deux fois
\item Un utilisateur déjà connecté ne peut se reconnecter
\item Lorsque le nombre de personnes maximum est atteint dans une activité crée, plus personne ne peut s'y inscrire
\item Une personne déjà inscrite ne peut pas se réinscrire
\item Un email doit contenir un '@'
\item \textbf{Liste de gestion des erreurs mise en place}
\item  
\item 
\end{itemize}
\end{description}

\newpage

\section{Conclusions}

\subsection{Résumé de la contribution}

\subsection{Difficultés rencontrées}
\begin{description}
\item\begin{itemize}
\item Création de sous choix
\item 
\item 
\item 
\item 
\item 
\end{itemize}
\end{description}
\subsection{Perspectives}





\newpage


\end{document}