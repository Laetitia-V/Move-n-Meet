\documentclass{article}

\usepackage[svgnames]{xcolor}
\usepackage{url}
\usepackage{xcolor}
\usepackage[utf8]{inputenc}
\usepackage[francais]{babel}
\usepackage{graphicx}

\usepackage{geometry}
\geometry{hmargin=2.5cm,vmargin=2.5cm}

\usepackage{fancyhdr}
\pagestyle{fancy}
\fancyhead[R]{Rapport de projet}
\fancyhead[L]{Aliénor Roussel, Laëtitia Viau, 
\newline Thibault Souvannavong, Abdellatif Ben Mlah}
\fancyfoot[C]{}

\renewcommand{\footrulewidth}{1pt}
\fancyfoot[C]{\textbf{Page \thepage}}

\renewcommand{\thesubsection}{\Alph{subsection}}
\renewcommand{\thesubsubsection}{\arabic{subsubsection}}

\setlength{\parindent}{0cm}
\setlength{\parskip}{1ex plus 0.5ex minus 0.2ex}
\newcommand{\hsp}{\hspace{20pt}}
\newcommand{\HRule}{\rule{\linewidth}{0.5mm}}

\begin{document}

\begin{titlepage}
  \begin{sffamily}
  \begin{center}

    \textsc{\LARGE Rapport de projet}\\[1cm]

    % Title
    \HRule \\[0.1cm]
 \includegraphics[scale=0.12]{logo.png}
    
    \HRule \\[1cm]
    \begin{center}
       \includegraphics[scale=2.2]{photo_couv.jpg}
          \\[0.5cm]
       \end{center} 
 
\Large{Aliénor Roussel, Laëtitia Viau, Thibault Souvannavong et Abdellatif Ben Mlah}

\vfill 
\vfill

    {\large Projet Licence 3 MIASHS 2017-2018}
\includegraphics[scale=0.2]{fac.png}
  \end{center}
  \end{sffamily}
\end{titlepage}

\tableofcontents

\newpage

\section{Introduction}

L'enjeux de notre projet est d'apporter une interface web permettant d’élaborer une liste d’activités à faire et des bons plans sur Montpellier et ses alentours. A travers ce site web, ses utilisateurs ont également la possibilité de proposer et/ou de rejoindre des activités de groupe. En effet, nous souhaitons offrir la possibilité aux utilisateurs de faire de nouvelles rencontres.\\
Pour cela, différents critères de recherche en fonction des préférences de l’utilisateur ont été mis en place.\\
Un utilisateur peut donc à la fois rechercher des idées d'activités et à la fois créer ou rejoindre des activités avec d'autres personnes.
Notre domaine d'application se limite à Montpellier et ses alentours afin de proposer des types d'activités variés dans le but de répondre aux demandes de chaque utilisateur.

\textbf{URL de notre projet : \url{https://github.com/Laetitia-V/Move-n-Meet}}

\section{Gestion de projet}

\begin{center}
{\textbf{Tableau de la contribution de chacun par tâche}}
\includegraphics[scale=0.5]{tache.png}
\end{center}

N'ayant pas encore eu les enseignements sur la création de page web dynamique au premier semestre nous avons fais une estimation de planning pour notre projet.
Il s'est avéré que ce planning a été grandement modifié. 
En effet, nous avons commencé la création des pages html et la création des feuilles de style CSS en même temps puisque étant donné que les deux étaient liés.
De plus, la conception de la base de donnée a été plus longue que prévu du fait du manque de données : beaucoup d'informations étaient manquantes dans les données que nous avions récoltées au premier semestre. Nous avons donc eu besoin de nous y adapter.\\
La quantité et la complexité des tables que nous avons du mettre en place pour avoir au final une base claire et logique a également participé à la durée de la tâche.\\
Il a par ailleurs fallu faire face à des contre-temps tel que le blocage de la faculté et donc le manque de compétences dans certaines branches de programmation. 
Nous avons donc dû faire des choix pour rendre le projet à temps et n'avons pas pu mettre en place le système de localisation sur carte des activités. \\
Aussi, nous avons décidé de remplacer la page des bons plans par nos activités coups de coeur sur la page d'accueil.
Également nous avons cherché à mettre en place certaines fonctionnalités du site grâce à des tutoriels internet suite au retard dans certaines matières (JavaScript). 


\section{Description technique}
\begin{description}
\item \textbf{Liste des technologies utilisées}
\begin{itemize}
\item Git
\item Php
\item Javascript
\item Mamp
\item Bluefish
\item LaTex
\item HTML
\item CSS

\end{itemize}
\end{description}


\begin{center}
{\textbf{Schéma de l'architecture de l'application}}
\includegraphics[scale=0.5]{archi.png}

{\textbf{Schéma de la base de données}}
\includegraphics[scale=0.5]{bd.png}
\end{center}

\section{Description fonctionnelle}

\begin{description}
\item \textbf{Liste des fonctionnalités de Move'n Meet}
\begin{itemize}
\item Connexion / Inscription sur le site
\item Déconnexion du site
\item Création d'une activité avec un nombre de personnes maximum
\item Recherche d'une idée d'activité à travers des critères et des sous-critères
\item Parcourir la liste des évènements temporaires
\item Parcourir une liste de nos coups de coeur
\item Rejoindre une activité déjà crée après s'être connecté
\item Se désinscrire d'une activité 
\item Poster des commentaires sur les activités
\item Consulter son profil
\item Modifier les informations de son profil
\item Voir le profil des autres utilisateurs
\item Utiliser la barre de recherche de la page d'accueil pour chercher une activité directement
\item Consulter son profil et ses activités de groupe futures lorsque l'on est connecté
\item  Envoyer et recevoir des messages entre les participants d'activités de groupe
\end{itemize}
\end{description}

\begin{description}
\item \textbf{Liste des fonctionnalités à mettre place}
\begin{itemize}
\item Une personne connectée peut laisser un commentaire sur les activités 
\item Lorsqu'il y a beaucoup d'activités retournées, faire en sorte qu'il y est une limite pour qu'elles apparaissent sur plusieurs pages
\item Faire en sorte que la page de recherche passe vérifie tous les cas possibles pour que l'utilisateur ait un résultat affiné avec sa recherche
\item Lorsque la date d'un évènement temporaire est passée, faire en sorte que l'évènement n'apparaisse plus

\end{itemize}
\end{description}

\begin{description}
\item \textbf{Liste de gestion des erreurs mise en place}
\begin{itemize}
\item Un utilisateur ne peut pas s'inscrire deux fois
\item Un utilisateur déjà connecté ne peut se reconnecter
\item Lorsque le nombre de personnes maximum est atteint dans une activité crée les suivants qui s'inscrivent sont mis en attente dans l'ordre d'inscription en attendant qu'une personne se désiste
\item Une personne déjà inscrite ne peut pas se réinscrire
\item Un email doit contenir un '@'
\item Pour créer une activité de groupe il faut que la date soit postérieure
\item Lors de l'inscription la date de naissance doit être antérieure au jour de l'inscription
\item Lors de l'inscription les deux mots de passe doivent être similaires
\item L'administrateur peut supprimer des sorties
\item Impossible de rejoindre deux fois la même sortie de groupe
\item Un champ doit être requis pour faire une recherche dans la barre de recherche
\item Lorsqu'un participant inscrit à une activité de groupe se désinscrit le premier de la liste d'attente est inscrit automatiquement
\item Un utilisateur ne peut pas créer un compte si son adresse ou son pseudo existe déjà dans la base de donnée
//
\end{itemize}
\end{description}

\begin{description}
\item \textbf{Liste de gestion des erreurs à mettre en place}
\begin{itemize}

\item Affiner les résultats de la barre de recherche
\item Faire en sorte qu'on puisse taper deux mots à la suite dans la barre de recherche
\item Affiner les résultats des activités
\item Après une recherche par type dans les activités de groupe pouvoir modifier la recherche et ne pas avoir à faire de retour en arrière
\item 

\end{itemize}
\end{description}

\newpage

\section{Conclusions}

\subsection{Résumé de la contribution}



\subsection{Difficultés rencontrées}
\begin{description}
\item\begin{itemize}
\item Création de sous choix
\item Mise à jour de la variable de session utilisateur
\item Création d'une barre de recherche
\item Résultats retournés de la barre de recherche
\item Le CSS n'est certaines fois pas pris en compte
\item Recherche par type des activités de groupe
\end{itemize}
\end{description}
\subsection{Perspectives}
\begin{description}
\item\begin{itemize}
\item Pour Trouver_activité nous envisageons le fait qu’un utilisateur puisse laisser un avis sur les activités et les évènements proposés sur le site. Ainsi selon la côte de popularité, les activités seraient plus ou moins mise en avant.
De plus, avec plus de moyen, il serait possible d’acheter des données plus fournies (Chambre de commerce) pour avoir un maximum d’informations sur les activités répertoriées sur le site. Cela permettrait de mettre plus de critères dans la recherche d’activités et pouvoir proposer l'activité qui correspond à la personne qui recherche.

\item
Au départ du projet, une page «Bons plans » était prévue. Cette page devait proposer des promotions permanentes mais aussi des tarifs avantageux pour les utilisateurs de notre site. Cependant, cela demande plus de temps et de moyens pour nouer des relations avec des partenaires et faire un système de bons de réduction électroniques pour les membres du site. 

\item Pour Activités de groupes, nous souhaitons mettre en place un système de rappel d’activités sur la boîte mail ou le téléphone des personnes concernées. 
\item La messagerie doit également être améliorée. 
\item Nous voulons ajouter un système de  « statut » pour chaque membre qui prend en compte le nombres de sorties effectuées, le nombre de sorties crées et le nombre de signalements qui a lieu par exemple lorsqu'un utilisateur à un mauvais comportement lors d’une sortie de groupe. 
\item Nous avons pour souhait de développer une application qui ne concernerait que les « activités de groupes » pour que les utilisateurs puissent s’inscrire directement depuis leur smartphone et être informés de tout imprévu.
\\
\item L'ergonomie du site peut également être amélioré grâce à l'optimisation de la navigation sur le site en fluidifiant les interactions entre l’utilisateur et les pages web. 
\item L'amélioration du CSS est également un objectif.
\item Mettre en place un système économique qui ne coûte rien aux utilisateurs tel que la publicité.
\end{itemize}
\end{description}
\newpage


\end{document}